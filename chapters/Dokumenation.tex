\section{Dokumentation}
\begin{itemize}[leftmargin=*]
    \item Die Dokumentation findet in den Header-Dateien statt.
    \item Dokumentation im Doxygen-Styl
    \item Jede Funktion/Methode muss dokumentiert sein.\\ 
          Private Methoden müssen mit @internal dokumentiert werden. Die Dokumentation darf weniger ausführlich sein, da sie nur für Wartungs- und Erweiterungszwecke verwendet wird.
    \item Public/Protected Felder müssen mit ///\textless{} Dokumentiert werden, Private Felder nicht.
    \item In den CPP-Dateien findet nur eine rudimentäre Implementierungsdokumentation statt. Diese soll “wenn” nötig die Implementierung erklären. Sie findet durch einzelne kurze Kommentare im Quellcode statt (Kein Doxygen-Styl)
\end{itemize}